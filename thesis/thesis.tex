\documentclass[a4paper,11pt]{report}

%\usepackage{UmUThesis}           % Standard English
\usepackage[noindent]{UmUThesis}  % Non indented English
%\usepackage[se]{UmUThesis}       % Swedish

\usepackage[utf8]{inputenc}
\usepackage{courier}              % Nicer fonts are used. (not necessary)
\usepackage{pslatex}              % Also nicer fonts. (not necessary)
%\usepackage{lmodern}             % Optional fonts. (not necessary)

%\usepackage{tabularx}
%\usepackage{graphicx}


\title{The title of your thesis}
\subtitle{If you have a subtitle}
\author{Willy Själin}
\supervisor{Lili Jiang, Associate Professor, Department of Computing Science, Umeå University}
\supervisore{Liubov Guseva and Elin Lidman, Tieto}
\examiner{Henrik Björklund, Associate Professor, Department of Computing Science, Umeå University}
\semester{Spring 2026}
\course{Degree Project in Computing Science and Engineering, 30 credits}
\education{Master of Science Programme in Computing Science and Engineering, 300 credits}

\graphicspath{{pictures/}}

\pagestyle{empty}

\begin{document}
\maketitle

\cleardoublepage


\begin{abstract}
An abstract is a short description (10-20 lines) of your thesis. Because on-line
search databases typically contain only abstracts, it is vital to write a
complete but concise description of your work to entice potential readers into
obtaining a copy of the full paper.

Although an abstract is brief, it should do almost as much work as the multi-page 
paper that follows it. Each chapter or section is typically a single sentence, 
but there is always room for creativity. In particular, parts may be merged or 
spread among a set of sentences.

\end{abstract}



\cleardoublepage

\chapter*{Acknowledgements}


\cleardoublepage

\tableofcontents
\cleardoublepage

\pagestyle{fancy}
\setcounter{page}{1}

\chapter{Introduction}

Software is a central part of everyday life. The mobile phone in a person's pocket contains more computational power than the computers that landed the first humans on the moon.
Early mobile phones, introduced in the 1970s, were designed with a single purpose: enabling mobile voice communication. At the time, most electronic devices had one well-defined task: 
mobile phones for calling, refrigerators for preserving food, and microwaves for reheating yesterday's leftovers. Today, it is common for such devices to offer far more functionality than originally intended.
Modern smartphones are equipped with more powerful processors, increased memory capacity, and advanced operating systems. As a result, users can manage their finances, find the closest restaurant, or
play games with a simple press of a finger \cite{mobile_phones}.

As software systems grow increasingly complex, the development process follows the same trend, incorporating activities such as planning, testing, integration and releases. To address the challenges
of complex software system, development activities are often performed more frequently. The method of continuous integration  has massively increased in popularity and is an
explicit recommendation of the practices of Extreme Programming \cite{continuous_software_engineering}. 

% Things to get in:
% CI/CD pipelines somewhere
% Eiffel events somewhere
% Something about anomaly detection. The introduction does not have to be that big.


\section{Motivation}
\section{Objective}
\section{Contributions}


\chapter{Background} % Maybe "Related works".
\section{Continuous Integration and Continuous Deployment}
\section{Eiffel framework}

\chapter{Method}
\chapter{Evaluation}
\chapter{Discussion}
\chapter{Conclusion}




\cleardoublepage

%
% In order to use the bibliography{base}-command you have to prepare a "database"
% base.bib in a certain format and then run the bibtex-command (a Unix-command)
% to create a file base.bbl which LaTeX uses to create References
%
\addcontentsline{toc}{chapter}{\bibname}
\bibliographystyle{plain}
\bibliography{base}

\appendix

\chapter{First Appendix}

If any.

\end{document}
