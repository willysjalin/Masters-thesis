\documentclass[10pt, titlepage, oneside, a4paper]{article}
\usepackage[T1]{fontenc}
\usepackage[utf8]{inputenc}
\usepackage[swedish]{babel}
\usepackage{amssymb, graphicx, fancyhdr, color, colortbl}
\usepackage{float}
\usepackage{listings,hyperref}
\usepackage{array}
\usepackage{tabu}
\usepackage{tabularx,ragged2e,booktabs,caption}
\usepackage{dirtree}
\usepackage{siunitx}
\hypersetup{
    colorlinks=true,
    linkcolor=black,
    filecolor=magenta,      
    urlcolor=blue,
    citecolor=red,
}
\usepackage[
backend=biber,
style=numeric,
sorting=none
]{biblatex}
\addbibresource{references.bib}
\addtolength{\textheight}{20mm}
\addtolength{\voffset}{-5mm}
 
% \Section ger mindre spillutrymme, använd dem om du vill
\newcommand{\Section}[1]{\section{#1}\vspace{-8pt}}
\newcommand{\Subsection}[1]{\vspace{-4pt}\subsection{#1}\vspace{-8pt}}
\newcommand{\Subsubsection}[1]{\vspace{-4pt}\subsubsection{#1}\vspace{-8pt}}
    
% appendices, \appitem och \appsubitem är för bilagor
\newcounter{appendixpage}
 
\newenvironment{appendices}{
    \setcounter{appendixpage}{\arabic{page}}
    \stepcounter{appendixpage}
}{
}
 
\newcommand{\appitem}[2]{
    \stepcounter{section}
    \addtocontents{toc}{\protect\contentsline{section}{\numberline{\Alph{section}}#1}{\arabic{appendixpage}}}
    \addtocounter{appendixpage}{#2}
}
 
\newcommand{\appsubitem}[2]{
    \stepcounter{subsection}
    \addtocontents{toc}{\protect\contentsline{subsection}{\numberline{\Alph{section}.\arabic{subsection}}#1}{\arabic{appendixpage}}}
    \addtocounter{appendixpage}{#2}
}
 
% Ändra de rader som behöver ändras
\def\typeofdoc{Project Diary}
\def\course{Master's thesis}
\def\courseCode{KURSKOD}
\def\pretitle{Obligatorisk Uppgift X}
\def\title{LABORATIONSNAMN}
\def\name{Willy Själin}
\def\username{willys}
\def\email{\username{}@cs.umu.se}
 
% Här börjar själva dokumentet
\begin{document}
 
    % skapar framsidan (om den inte duger: gör helt enkelt en egen)
    \begin{titlepage}
        \thispagestyle{empty}
        \begin{large}
            \begin{tabular}{@{}p{\textwidth}@{}}
                \textbf{\typeofdoc \hfill \today} \\
            \end{tabular}
        \end{large}
        \vspace{20mm}
        \begin{center}
            \huge{\textbf{\course}}\\
            \vspace{15mm}
            \begin{large}
                \begin{tabular}{ll}
                    \textbf{Namn} & \name \\
                    \textbf{E-mail} & \texttt{\email} \\
                \end{tabular}
            \end{large}
            \vfill
        \end{center}
    \end{titlepage}
 
 
    % fixar sidfot
    \lfoot{\footnotesize{\name, \email}}
    \rfoot{\footnotesize{\today}}
    \lhead{\sc\footnotesize\title}
    \rhead{\nouppercase{\sc\footnotesize\leftmark}}
    \pagestyle{fancy}
    \renewcommand{\headrulewidth}{0.2pt}
    \renewcommand{\footrulewidth}{0.2pt}
 
    % skapar innehållsförteckning.
    % Tänk på att köra latex 2ggr för att uppdatera allt
    \pagenumbering{roman}
    
    % och lägger in en sidbrytning
 
    \pagenumbering{arabic}
 
    % i Sverige har vi normalt inget indrag vid nytt stycke
    \setlength{\parindent}{0pt}
    % men däremot lite mellanrum
    \setlength{\parskip}{10pt}
 
    % lägger in rubrik (finns \section, men då får man mycket spillutrymme)
    \section*{Week 1}
    The first week of the thesis project began with getting introduced to the office and receiving a laptop that I could work on. I had the project specification finished before the course start,
    but the research questions were very vaguely stated and the AI-techniques was not even mentioned due to time had to be spent on the exams prior to the thesis course. I set up the repository, installed
    the necessary programs and began with the Gantt-chart. The majority of time spent on week 1 was to get familiar with the Eiffel-framework which is one of the most central part of the project. I mailed 
    my internal supervisor and we are going have a meeting in the near future to talk about the uncertainties of the specification.


    \section*{Week 2}
    During the second week of the project, a majority of the time has been spent on finding relative work and literature study. 
    An initial meeting with my internal supervisor has been done and it was good. The results of the meeting was to leave the research
    questions somewhat open until I have some more knowledge about the subject and translated it into the thesis. This week has not been the
    most productive week as I have had personal things in the way everyday. I'm hoping that the coming weeks will be better. 

    The next meeting is scheduled for February 11th.
    
    \section*{Week 3}
    I did the following ...
    \section*{Week 4}
    I did the following ...
    \section*{Week 5}
    I did the following ...
    \section*{Week 6}
    I did the following ...
    \section*{Week 7}
    I did the following ...
    \section*{Week 8}
    I did the following ...
    \section*{Week 9}
    I did the following ...
    \section*{Week 10}
    I did the following ...
    \section*{Week 11}
    I did the following ...
    \section*{Week 12}
    I did the following ...
    \section*{Week 13}
    I did the following ...
    \section*{Week 14}
    I did the following ...
    \section*{Week 15}
    I did the following ...
    \section*{Week 16}
    I did the following ...
    \section*{Week 17}
    I did the following ...
    \section*{Week 18}
    I did the following ...
    \section*{Week 19}
    I did the following ...
    \section*{Week 20}
    I did the following ...

 
\end{document}